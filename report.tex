% !TEX TS-program = xelatex
%
\documentclass{report}

  \usepackage[english]{babel}
  \usepackage{parskip}
  \usepackage{fontspec}
  \defaultfontfeatures{Ligatures=TeX}

  \title{Process Calculi for Concurrency}
  \author{Markus Reiter, Michael Kaltschmid}

  \begin{document}
  \maketitle
  \tableofcontents

  \section{Introduction}
  \subsection{What is a Process Calculus?}
  A Process Calculus is basically an approach for formally modelling concurrent systems. \\
  Furthermore it is a tool for high-level description of interactions, communications and synchronizations between processes. \\
  It also provides algebraic laws to allow analyzing and transforming process descriptions and permits formal reasoning about equivalences between processes (e.g., using bisimulation) \\

  \subsection{Focus}
  There are various forms of Process Calculi like $ACP$, $CCS$, $CSP$, $Join-Calculus$, $\mu-Calculus$, $PEPA$ or $\pi-Calculus$. However we are only going to focus on $ACP$ and $\mu-Calculus$. Specifically on $\mu-Calculus$ since it is the base for $mCRL2$.

  \end{document}
